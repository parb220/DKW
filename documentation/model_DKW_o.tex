%2multibyte Version: 5.50.0.2953 CodePage: 1252

\documentclass{article}
%%%%%%%%%%%%%%%%%%%%%%%%%%%%%%%%%%%%%%%%%%%%%%%%%%%%%%%%%%%%%%%%%%%%%%%%%%%%%%%%%%%%%%%%%%%%%%%%%%%%%%%%%%%%%%%%%%%%%%%%%%%%%%%%%%%%%%%%%%%%%%%%%%%%%%%%%%%%%%%%%%%%%%%%%%%%%%%%%%%%%%%%%%%%%%%%%%%%%%%%%%%%%%%%%%%%%%%%%%%%%%%%%%%%%%%%%%%%%%%%%%%%%%%%%%%%
\usepackage{amsfonts}
\usepackage{amsmath}

\setcounter{MaxMatrixCols}{10}
%TCIDATA{OutputFilter=LATEX.DLL}
%TCIDATA{Version=5.50.0.2953}
%TCIDATA{Codepage=1252}
%TCIDATA{<META NAME="SaveForMode" CONTENT="1">}
%TCIDATA{BibliographyScheme=Manual}
%TCIDATA{Created=Tuesday, May 06, 2008 16:53:02}
%TCIDATA{LastRevised=Thursday, June 11, 2015 17:59:44}
%TCIDATA{<META NAME="GraphicsSave" CONTENT="32">}
%TCIDATA{<META NAME="DocumentShell" CONTENT="Standard LaTeX\Blank - Standard LaTeX Article">}
%TCIDATA{CSTFile=40 LaTeX article.cst}
%TCIDATA{ComputeDefs=
%$n$
%}


\newtheorem{theorem}{Theorem}
\newtheorem{acknowledgement}[theorem]{Acknowledgement}
\newtheorem{algorithm}[theorem]{Algorithm}
\newtheorem{axiom}[theorem]{Axiom}
\newtheorem{case}[theorem]{Case}
\newtheorem{claim}[theorem]{Claim}
\newtheorem{conclusion}[theorem]{Conclusion}
\newtheorem{condition}[theorem]{Condition}
\newtheorem{conjecture}[theorem]{Conjecture}
\newtheorem{corollary}[theorem]{Corollary}
\newtheorem{criterion}[theorem]{Criterion}
\newtheorem{definition}[theorem]{Definition}
\newtheorem{example}[theorem]{Example}
\newtheorem{exercise}[theorem]{Exercise}
\newtheorem{lemma}[theorem]{Lemma}
\newtheorem{notation}[theorem]{Notation}
\newtheorem{problem}[theorem]{Problem}
\newtheorem{proposition}[theorem]{Proposition}
\newtheorem{remark}[theorem]{Remark}
\newtheorem{solution}[theorem]{Solution}
\newtheorem{summary}[theorem]{Summary}
\newenvironment{proof}[1][Proof]{\noindent\textbf{#1.} }{\ \rule{0.5em}{0.5em}}
\setlength{\oddsidemargin}{.25in}
\setlength{\evensidemargin}{.05in}
\setlength{\textwidth}{6.0in}
\setlength{\textheight}{8.2in}
\renewcommand{\topmargin}{0.0in}
\renewcommand{\baselinestretch}{1.0}
\input{tcilatex}
\makeatletter
\def\@biblabel#1{\hspace*{-\labelsep}}
\makeatother

\begin{document}

\title{Documentation: MODEL "DKW\_o"}
\author{Nikolay Gospodinov and Bin Wei}
\maketitle

\section{Model}

Suppose that there are a vector of four latent variables $x_{t}=\left(
x_{1t},x_{2t},x_{3t}\right) ^{\prime }$ and $v_{t}$\textbf{\ }that drive
nominal and real yields as well as inflation. Their dynamics under the
physical measure is%
\begin{equation}
dx_{t}=\mathcal{K}\left( \mu -x_{t}\right) dt+\Sigma dW_{x,t}
\end{equation}%
The nominal pricing kernel takes the form%
\begin{equation}
dM_{t}^{N}/M_{t}^{N}=-r_{t}^{N}dt-\Lambda _{x,t}^{N\prime }dW_{x,t}
\end{equation}%
where the nominal short rate is%
\begin{equation}
r_{t}^{N}=\rho _{0}^{N}+\rho _{x}^{N\prime }x_{t}
\end{equation}%
and the vector of prices of risk is given by\qquad 
\begin{equation*}
\Lambda _{x,t}^{N}=\lambda _{0}^{N}+\lambda _{x}^{N}x_{t}
\end{equation*}%
Let $q_{t}\equiv \log Q_{t}$ denote the log price level. The price level
evolves as follows: 
\begin{equation}
dq_{t}=\pi _{t}dt+\sigma _{q}^{\prime }dW_{x,t}+\sigma _{\bot }dW_{\bot ,t}
\end{equation}%
where the instantaneous expected inflation rate is given by 
\begin{equation}
\pi _{t}=\rho _{0}^{\pi }+\rho _{x}^{\pi \prime }x_{t}
\end{equation}

\textbf{Real Pricing Kernel.} Based on the observation $%
M_{t}^{R}=M_{t}^{N}Q_{t}$, we have%
\begin{eqnarray*}
dM_{t}^{R}/M_{t}^{R} &=&dM_{t}^{N}/M_{t}^{N}+dQ_{t}/Q_{t}+\left(
dM_{t}^{N}/M_{t}^{N}\right) \cdot \left( dQ_{t}/Q_{t}\right)  \\
&=&-r_{t}^{N}dt-\Lambda _{x,t}^{N\prime }dW_{x,t}+\left[ \pi _{t}+\frac{1}{2}%
\left( \sigma _{q}^{\prime }\sigma _{q}+\left( \sigma _{q}^{\bot }\right)
^{2}\right) -\sigma _{q}^{\prime }\Lambda _{x,t}^{N}\right] dt+\sigma
_{q}^{\prime }dW_{x,t} \\
&\equiv &-r_{t}^{R}dt-\Lambda _{x,t}^{R\prime }dW_{x,t}
\end{eqnarray*}%
where 
\begin{eqnarray*}
r_{t}^{R} &=&r_{t}^{N}-\left[ \pi _{t}+\frac{1}{2}\left( \sigma _{q}^{\prime
}\sigma _{q}+\left( \sigma _{q}^{\bot }\right) ^{2}\right) \right] +\sigma
_{q}^{\prime }\Lambda _{x,t}^{N}\equiv \rho _{0}^{R}+\rho _{x}^{R\prime
}x_{t} \\
\Lambda _{x,t}^{R} &=&\Lambda _{x,t}^{N}-\sigma _{q}\equiv \lambda
_{0}^{R}+\lambda _{x}^{R\prime }x_{t}
\end{eqnarray*}%
and%
\begin{eqnarray*}
\rho _{0}^{R} &=&\rho _{0}^{N}-\rho _{0}^{\pi }-\frac{1}{2}\left( \sigma
_{q}^{\prime }\sigma _{q}+\left( \sigma _{q}^{\bot }\right) ^{2}\right)
+\lambda _{0}^{\prime }\sigma _{q} \\
\rho _{x}^{R} &=&\rho _{x}^{N}-\rho _{x}^{\pi }+\lambda _{x}^{N\prime
}\sigma _{q}
\end{eqnarray*}%
and 
\begin{equation*}
\lambda _{0}^{R}=\lambda _{0}^{N}-\sigma _{q},\text{ and }\lambda
_{x}^{R}=\lambda _{x}^{N}
\end{equation*}

\bigskip 

\textbf{Risk-Neutral Measure.} The Radon-Nikodym derivative of the risk
neutral measure $\mathbb{P}^{\ast }$ with respect to the physical measure $%
\mathbb{P}$ is given by%
\begin{equation}
\left( \frac{d\mathbb{P}^{\ast }}{d\mathbb{P}}\right) _{t,T}=\exp \left[ -%
\frac{1}{2}\int_{t}^{T}\Lambda _{s}^{N\prime }\Lambda
_{s}^{N}ds-\int_{t}^{T}\Lambda _{s}^{N\prime }dW_{s}\right] 
\end{equation}%
where $\Lambda _{t}^{N}\equiv \Lambda _{x,t}^{N}$ and $W_{t}\equiv W_{x,t}$.
Then,By the Girsanov theorem, $dW_{t}^{\ast }=dW_{t}+\Lambda _{t}^{N}dt$ is
a standard Brownian motion under the risk-neutral probability measure $%
\mathbb{P}^{\ast }$. It implies that under the risk neutral measure, the
dynamics is given by%
\begin{eqnarray*}
dx_{t} &=&\mathcal{K}\left( \mu -x_{t}\right) dt+\Sigma dW_{x,t} \\
&=&\mathcal{K}\left( \mu -x_{t}\right) dt+\Sigma \left( dW_{x,t}^{\ast
}-\Lambda _{x,t}^{N}dt\right)  \\
&=&\left[ \left( \mathcal{K}\mu -\Sigma \lambda _{0}^{N}\right) -\left( 
\mathcal{K}+\Sigma \lambda _{x}^{N}\right) x_{t}\right] dt+\Sigma
dW_{x,t}^{\ast } \\
&\equiv &\mathcal{K}^{\ast }\left( \mu ^{\ast }-x_{t}\right) dt+\Sigma
dW_{x,t}^{\ast },
\end{eqnarray*}%
and%
\begin{eqnarray*}
dq_{t} &=&\pi _{t}dt+\sigma _{q}^{\prime }dW_{x,t}+\sqrt{v_{t}}dW_{\bot ,t}
\\
&=&\left( \rho _{0}^{\pi }+\rho _{x}^{\pi \prime }x_{t}\right) dt+\sigma
_{q}^{\prime }\left( dW_{x,t}^{\ast }-\Lambda _{x,t}^{N}dt\right) +\sigma
_{q}^{\bot }dW_{\bot ,t}^{\ast } \\
&=&\left( \rho _{0}^{\pi }+\rho _{x}^{\pi \prime }x_{t}-\sigma _{q}^{\prime
}\left( \lambda _{0}^{N}+\lambda _{x}^{N}x_{t}\right) \right) dt+\sigma
_{q}^{\prime }dW_{x,t}^{\ast }+\sigma _{q}^{\bot }dW_{\bot ,t}^{\ast } \\
&\equiv &\left( \rho _{0}^{\pi \ast }+\rho _{x}^{\pi \ast \prime
}x_{t}\right) dt+\sigma _{q}^{\prime }dW_{x,t}^{\ast }+\sigma _{q}^{\bot
}dW_{\bot ,t}^{\ast },
\end{eqnarray*}%
where 
\begin{eqnarray*}
\mathcal{K}^{\ast } &=&\mathcal{K+}\Sigma \lambda _{x}^{N} \\
\mathcal{K}^{\ast }\mu ^{\ast } &=&\mathcal{K}\mu \mathcal{-}\Sigma \lambda
_{0}^{N} \\
\pi _{t}^{\ast } &=&\rho _{0}^{\pi \ast }+\rho _{x}^{\pi \ast \prime }x_{t}
\\
\rho _{0}^{\pi \ast } &=&\rho _{0}^{\pi }-\lambda _{0}^{N\prime }\sigma _{q}
\\
\rho _{x}^{\pi \ast } &=&\rho _{x}^{\pi }-\lambda _{x}^{N\prime }\sigma _{q}
\end{eqnarray*}

\textbf{Forward Measure.} The Radon-Nikodym derivative of the forward
measure $\mathbb{\tilde{P}}$ with respect to the risk neutral measure $%
\mathbb{P}^{\ast }$ is given by%
\begin{equation}
\left( \frac{d\mathbb{\tilde{P}}}{d\mathbb{P}^{\ast }}\right) _{t,T}=\frac{%
\exp \left( -\int_{t}^{T}r_{s}^{N}ds\right) }{P_{t,\tau }^{N}}
\end{equation}%
and%
\begin{equation*}
\Psi _{t}\equiv E_{t}^{\mathbb{P}^{\ast }}\left[ \left( \frac{d\mathbb{%
\tilde{P}}}{d\mathbb{P}^{\ast }}\right) _{0,T}\right] =E_{t}^{\mathbb{P}%
^{\ast }}\left[ \frac{\exp \left( -\int_{0}^{T}r_{s}^{N}ds\right) }{%
P_{0,T}^{N}}\right] =\frac{P_{t,\tau }^{N}}{P_{0,T}^{N}}\exp \left(
-\int_{0}^{t}r_{s}^{N}ds\right) 
\end{equation*}%
We have%
\begin{equation*}
d\Psi _{t}=\frac{\exp \left( -\int_{0}^{t}r_{s}^{N}ds\right) }{P_{0,T}^{N}}%
\left[ dP_{t,\tau }^{N}-r_{t}^{N}P_{t,\tau }^{N}dt\right] =\Psi _{t}B_{\tau
}^{N\prime }\Sigma dW_{x,t}
\end{equation*}%
By Girsanov's Theorem, we have%
\begin{equation*}
d\tilde{W}_{t}=dW_{t}^{\ast }-\frac{d\Psi _{t}}{\Psi _{t}}\cdot dW_{t}^{\ast
}
\end{equation*}%
or%
\begin{equation*}
d\tilde{W}_{x,t}=dW_{t}^{\ast }-\Sigma ^{\prime }B_{\tau }^{N}dt
\end{equation*}%
The dynamics of the state variables under the risk neutral measure is given
by%
\begin{eqnarray*}
dx_{t} &=&\mathcal{K}^{\ast }\left( \mu ^{\ast }-x_{t}\right) +\Sigma
dW_{x,t}^{\ast } \\
dq_{t} &=&\left( \rho _{0}^{\pi \ast }+\rho _{x}^{\pi \ast \prime
}x_{t}+\rho _{v}^{\pi \ast }v_{t}\right) dt+\sigma _{q}^{\prime
}dW_{x,t}^{\ast }+\sigma _{q}^{\bot }dW_{\bot ,t}^{\ast }.
\end{eqnarray*}%
Therefore,%
\begin{eqnarray*}
dx_{t} &=&\mathcal{K}^{\ast }\left( \mu ^{\ast }-x_{t}\right) dt+\Sigma
\left( d\tilde{W}_{x,t}+\Sigma ^{\prime }B_{\tau }^{N}dt\right)  \\
&=&\left( \mathcal{K}^{\ast }\mu ^{\ast }+\Sigma \Sigma ^{\prime }B_{\tau
}^{N}-\mathcal{K}^{\ast }x_{t}\right) dt+\Sigma d\tilde{W}_{x,t} \\
&\equiv &\mathcal{\tilde{K}}\left( \tilde{\mu}-x_{t}\right) dt+\Sigma d%
\tilde{W}_{x,t},
\end{eqnarray*}%
and%
\begin{eqnarray*}
dq_{t} &=&\left( \rho _{0}^{\pi \ast }+\rho _{x}^{\pi \ast \prime
}x_{t}\right) dt+\sigma _{q}^{\prime }dW_{x,t}^{\ast }+\sigma _{q}^{\bot
}dW_{\bot ,t}^{\ast } \\
&=&\left( \left[ \rho _{0}^{\pi \ast }+\sigma _{q}^{\prime }\Sigma ^{\prime
}B_{\tau }^{N}\right] +\rho _{x}^{\pi \ast \prime }x_{t}\right) dt+\sigma
_{q}^{\prime }d\tilde{W}_{x,t}+\sigma _{q}^{\bot }d\tilde{W}_{\bot ,t} \\
&\equiv &\left( \widetilde{\rho }_{0}^{\pi }+\widetilde{\rho }_{x}^{\pi
\prime }x_{t}\right) dt+\sigma _{q}^{\prime }d\tilde{W}_{x,t}+\sigma
_{q}^{\bot }d\tilde{W}_{\bot ,t}
\end{eqnarray*}%
where 
\begin{eqnarray*}
\mathcal{\tilde{K}} &=&\mathcal{K}^{\ast }=\mathcal{K+}\Sigma \lambda
_{x}^{N} \\
\mathcal{\tilde{K}}\tilde{\mu} &=&\mathcal{K}^{\ast }\mu ^{\ast }+\Sigma
\Sigma ^{\prime }B_{\tau }^{N}=\mathcal{K}\mu \mathcal{-}\Sigma \lambda
_{0}^{N}+\Sigma \Sigma ^{\prime }B_{\tau }^{N} \\
\tilde{\pi}_{t} &=&\widetilde{\rho }_{0}^{\pi }+\widetilde{\rho }_{x}^{\pi
\prime }x_{t} \\
\widetilde{\rho }_{0}^{\pi } &=&\rho _{0}^{\pi \ast }+\sigma _{q}^{\prime
}\Sigma ^{\prime }B_{\tau }^{N}=\rho _{0}^{\pi }-\lambda _{0}^{N\prime
}\sigma _{q}+\sigma _{q}^{\prime }\Sigma ^{\prime }B_{\tau }^{N} \\
\widetilde{\rho }_{x}^{\pi } &=&\rho _{x}^{\pi \ast }=\rho _{x}^{\pi
}-\lambda _{x}^{N\prime }\sigma _{q}
\end{eqnarray*}

\textbf{Inflation Options.} Under the put-call parity, we can show that 
\begin{equation*}
E_{t}^{\mathbb{\tilde{P}}}\left[ \frac{Q_{t+\tau }}{Q_{t}}\right] =\frac{%
P_{t,\tau ,K}^{CAP}-P_{t,\tau ,K}^{FLO}}{P_{t,\tau }^{N}}+\left( 1+K\right)
^{\tau }
\end{equation*}%
Note that 
\begin{eqnarray*}
\frac{1}{\tau }\log E_{t}^{\mathbb{\tilde{P}}}\left[ \frac{Q_{t+\tau }}{Q_{t}%
}\right]  &=&\frac{1}{\tau }\left( E_{t}^{\mathbb{\tilde{P}}}\left[
q_{t+\tau }-q_{t}\right] +\frac{1}{2}Var_{t}^{\mathbb{\tilde{P}}}\left[
q_{t+\tau }-q_{t}\right] \right)  \\
&\equiv &\mathcal{IE}_{t,\tau }+\frac{1}{2}\mathcal{IU}_{t,\tau }
\end{eqnarray*}

We first derive $\mathcal{IE}_{t,\tau }\equiv \frac{1}{\tau }E_{t}^{\mathbb{%
\tilde{P}}}\left[ q_{t+\tau }-q_{t}\right] $. Note 
\begin{eqnarray*}
q_{t+\tau }-q_{t} &=&\int_{t}^{t+\tau }\left[ \left( \widetilde{\rho }%
_{0}^{\pi }+\widetilde{\rho }_{x}^{\pi \prime }x_{s}\right) ds+\sigma
_{q}^{\prime }d\widetilde{W}_{x,s}+\sigma _{q}^{\bot }d\widetilde{W}_{\bot
,s}\right]  \\
x_{s} &=&\widetilde{\mu }+\exp \left( -\widetilde{\mathcal{K}}\left(
s-t\right) \right) \left( x_{t}-\widetilde{\mu }\right) +\int_{t}^{s}\exp
\left( -\widetilde{\mathcal{K}}\left( s-u\right) \right) \Sigma d\widetilde{W%
}_{x,u}
\end{eqnarray*}%
We have%
\begin{eqnarray*}
\frac{1}{\tau }\int_{t}^{t+\tau }E_{t}^{\mathbb{\tilde{P}}}\left[ x_{s}%
\right] ds &=&\frac{1}{\tau }\int_{t}^{t+\tau }\left[ \widetilde{\mu }+\exp
\left( -\widetilde{\mathcal{K}}\left( s-t\right) \right) \left( x_{t}-%
\widetilde{\mu }\right) \right] ds \\
&=&\widetilde{\mu }+\left( \mathcal{\tilde{K}}\tau \right) ^{-1}\left(
I-\exp \left( -\mathcal{\tilde{K}}\tau \right) \right) \left( x_{t}-%
\widetilde{\mu }\right)  \\
&\equiv &\tilde{a}_{\tau }^{x}+\tilde{b}_{\tau }^{x}x_{t}
\end{eqnarray*}%
where $\tilde{a}_{\tau }^{x}=\left( I-\tilde{b}_{\tau }^{x}\right) \tilde{\mu%
}$ and $\tilde{b}_{\tau }^{x}=\left( \mathcal{\tilde{K}}\tau \right)
^{-1}\left( I-\exp \left( -\mathcal{\tilde{K}}\tau \right) \right) $.
Therefore,%
\begin{eqnarray*}
\mathcal{IE}_{t,\tau } &\equiv &\frac{1}{\tau }E_{t}^{\mathbb{\tilde{P}}}%
\left[ q_{t+\tau }-q_{t}\right] =\widetilde{\rho }_{0}^{\pi }+\widetilde{%
\rho }_{x}^{\pi \prime }\left( \tilde{a}_{\tau }^{x}+\tilde{b}_{\tau
}^{x}x_{t}\right)  \\
&\equiv &\tilde{a}_{\tau }^{\pi }+\tilde{b}_{\tau }^{\pi \prime }x_{t}
\end{eqnarray*}%
where $\tilde{a}_{\tau }^{\pi }=\widetilde{\rho }_{0}^{\pi }+\widetilde{\rho 
}_{x}^{\pi \prime }\tilde{a}_{\tau }^{x}$ and $\tilde{b}_{\tau }^{\pi }=%
\tilde{b}_{\tau }^{x\prime }\widetilde{\rho }_{x}^{\pi }$. 

Next we derive $\mathcal{IU}_{t,\tau }\equiv \frac{1}{\tau }Var_{t}^{\mathbb{%
\tilde{P}}}\left[ q_{t+\tau }-q_{t}\right] $. Note that:\ 
\begin{eqnarray*}
Var_{t}^{\mathbb{\tilde{P}}}\left[ q_{t+\tau }-q_{t}\right]  &=&E_{t}^{%
\mathbb{\tilde{P}}}\left[ \left( \left[ q_{t+\tau }-q_{t}\right] -E_{t}^{%
\mathbb{\tilde{P}}}\left[ q_{t+\tau }-q_{t}\right] \right) ^{2}\right]  \\
&=&E_{t}^{\mathbb{\tilde{P}}}\left[ \left( \int_{t}^{t+\tau }\left[ 
\widetilde{\rho }_{x}^{\pi \prime }\left( x_{s}-E_{t}^{\mathbb{\tilde{P}}}%
\left[ x_{s}\right] \right) ds+\sigma _{q}^{\prime }d\widetilde{W}%
_{x,s}+\sigma _{q}^{\bot }d\widetilde{W}_{\bot ,s}\right] \right) ^{2}\right]
\\
&=&E_{t}^{\mathbb{\tilde{P}}}\left[ \left( \int_{t}^{t+\tau }\left[ 
\widetilde{\rho }_{x}^{\pi \prime }\int_{t}^{s}\exp \left( -\widetilde{%
\mathcal{K}}\left( s-u\right) \right) \Sigma d\widetilde{W}%
_{x,u}ds+\int_{t}^{t+\tau }\left[ \sigma _{q}^{\prime }d\widetilde{W}%
_{x,s}+\sigma _{q}^{\bot }d\widetilde{W}_{\bot ,s}\right] \right] \right)
^{2}\right]  \\
&=&E_{t}^{\mathbb{\tilde{P}}}\left[ \left( \widetilde{\rho }_{x}^{\pi \prime
}\int_{t}^{t+\tau }\int_{t}^{s}\exp \left( -\widetilde{\mathcal{K}}\left(
s-u\right) \right) \Sigma d\widetilde{W}_{x,u}ds+\int_{t}^{t+\tau }\left[
\sigma _{q}^{\prime }d\widetilde{W}_{x,s}+\sigma _{q}^{\bot }d\widetilde{W}%
_{\bot ,s}\right] \right) ^{2}\right]  \\
&=&E_{t}^{\mathbb{\tilde{P}}}\left[ \left( \widetilde{\rho }_{x}^{\pi \prime
}\int_{t}^{t+\tau }\int_{u}^{t+\tau }\exp \left( -\widetilde{\mathcal{K}}%
\left( s-u\right) \right) ds\Sigma d\widetilde{W}_{x,u}+\int_{t}^{t+\tau }%
\left[ \sigma _{q}^{\prime }d\widetilde{W}_{x,s}+\sigma _{q}^{\bot }d%
\widetilde{W}_{\bot ,s}\right] \right) ^{2}\right]  \\
&=&E_{t}^{\mathbb{\tilde{P}}}\left[ \left( \widetilde{\rho }_{x}^{\pi \prime
}\int_{t}^{t+\tau }\left( \widetilde{\mathcal{K}}\right) ^{-1}\left( I-\exp
\left( -\widetilde{\mathcal{K}}\left( t+\tau -s\right) \right) \right)
\Sigma d\widetilde{W}_{x,s}+\int_{t}^{t+\tau }\left[ \sigma _{q}^{\prime }d%
\widetilde{W}_{x,s}+\sigma _{q}^{\bot }d\widetilde{W}_{\bot ,s}\right]
\right) ^{2}\right]  \\
&\equiv &E_{t}^{\mathbb{\tilde{P}}}\left[ \left( \int_{t}^{t+\tau }\left[
\left( \widehat{\sigma }_{q}-\Sigma ^{\prime }\exp \left( -\widetilde{%
\mathcal{K}}^{\prime }\left( t+\tau -s\right) \right) \left( \widetilde{%
\mathcal{K}}^{\prime }\right) ^{-1}\widetilde{\rho }_{x}^{\pi }\right)
^{\prime }d\widetilde{W}_{x,s}+\sigma _{q}^{\bot }d\widetilde{W}_{\bot ,s}%
\right] \right) ^{2}\right]  \\
&=&\int_{t}^{t+\tau }\left( \widehat{\sigma }_{q}-\Sigma ^{\prime }e^{-%
\mathcal{K}^{\prime }\left( t+\tau -s\right) }\left( \widetilde{\mathcal{K}}%
^{\prime }\right) ^{-1}\rho _{x}^{\pi }\right) ^{\prime }\left( \widehat{%
\sigma }_{q}-\Sigma ^{\prime }\exp \left( -\widetilde{\mathcal{K}}^{\prime
}\left( t+\tau -s\right) \right) \left( \widetilde{\mathcal{K}}^{\prime
}\right) ^{-1}\widetilde{\rho }_{x}^{\pi }\right) ds+\left( \sigma
_{q}^{\bot }\right) ^{2}\tau  \\
&=&\left( \widehat{\sigma }_{q}^{\prime }\widehat{\sigma }_{q}+\left( \sigma
_{q}^{\bot }\right) ^{2}\right) \tau -2\widehat{\sigma }_{q}^{\prime }\Sigma
^{\prime }\left[ \int_{t}^{t+\tau }\exp \left( -\widetilde{\mathcal{K}}%
^{\prime }\left( t+\tau -s\right) \right) ds\right] \left( \widetilde{%
\mathcal{K}}^{\prime }\right) ^{-1}\widetilde{\rho }_{x}^{\pi } \\
&&+\rho _{x}^{\pi \prime }\left( \widetilde{\mathcal{K}}\right) ^{-1}\left[
\int_{t}^{t+\tau }\exp \left( -\widetilde{\mathcal{K}}\left( t+\tau
-s\right) \right) \Sigma \Sigma ^{\prime }\exp \left( -\widetilde{\mathcal{K}%
}^{\prime }\left( t+\tau -s\right) \right) ds\right] \left( \widetilde{%
\mathcal{K}}^{\prime }\right) ^{-1}\widetilde{\rho }_{x}^{\pi }
\end{eqnarray*}%
where 
\begin{equation*}
\widehat{\sigma }_{q}^{\prime }=\sigma _{q}^{\prime }+\widetilde{\rho }%
_{x}^{\pi \prime }\left( \mathcal{K}\right) ^{-1}\Sigma \text{ or }\widehat{%
\sigma }_{q}=\sigma _{q}+\Sigma ^{\prime }\left( \mathcal{K}^{\prime
}\right) ^{-1}\widetilde{\rho }_{x}^{\pi }
\end{equation*}%
and if we denote $\widetilde{\Omega }_{\tau }^{x}\equiv \int_{t}^{t+\tau
}\exp \left( -\widetilde{\mathcal{K}}\left( t+\tau -s\right) \right) \Sigma
\Sigma ^{\prime }\exp \left( -\widetilde{\mathcal{K}}^{\prime }\left( t+\tau
-s\right) \right) ds=\int_{0}^{\tau }\exp \left( -\widetilde{\mathcal{K}}%
s\right) \Sigma \Sigma ^{\prime }\exp \left( -\widetilde{\mathcal{K}}%
^{\prime }s\right) ds$, then 
\begin{equation*}
vec\left( \widetilde{\Omega }_{\tau }^{x}\right) =-\left[ \left( \widetilde{%
\mathcal{K}}\otimes I\right) +\left( I\otimes \widetilde{\mathcal{K}}\right) %
\right] ^{-1}vec\left( \exp \left( -\widetilde{\mathcal{K}}\tau \right)
\Sigma \Sigma ^{\prime }\exp \left( -\widetilde{\mathcal{K}}^{\prime }\tau
\right) -\Sigma \Sigma ^{\prime }\right) 
\end{equation*}

Therefore, 
\begin{eqnarray*}
\mathcal{IU}_{t,\tau } &\equiv &\frac{1}{\tau }Var_{t}^{\mathbb{\tilde{P}}}%
\left[ q_{t+\tau }-q_{t}\right]  \\
&=&\left( \widehat{\sigma }_{q}^{\prime }\widehat{\sigma }_{q}+\left( \sigma
_{q}^{\bot }\right) ^{2}\right) -2\widehat{\sigma }_{q}^{\prime }\Sigma
^{\prime }\left[ \frac{1}{\tau }\int_{t}^{t+\tau }\exp \left( -\widetilde{%
\mathcal{K}}^{\prime }\left( t+\tau -s\right) \right) ds\right] \left( 
\widetilde{\mathcal{K}}^{\prime }\right) ^{-1}\widetilde{\rho }_{x}^{\pi } \\
&&+\rho _{x}^{\pi \prime }\left( \widetilde{\mathcal{K}}\right) ^{-1}\left( 
\frac{1}{\tau }\widetilde{\Omega }_{\tau }^{x}\right) \left( \widetilde{%
\mathcal{K}}^{\prime }\right) ^{-1}\widetilde{\rho }_{x}^{\pi } \\
&=&\left( \widehat{\sigma }_{q}^{\prime }\widehat{\sigma }_{q}+\left( \sigma
_{q}^{\bot }\right) ^{2}\right) -2\widehat{\sigma }_{q}^{\prime }\Sigma
^{\prime }\left( \widetilde{\mathcal{K}}^{\prime }\tau \right) ^{-1}\left(
I-\exp \left( -\widetilde{\mathcal{K}}^{\prime }\tau \right) \right) \left( 
\widetilde{\mathcal{K}}^{\prime }\right) ^{-1}\widetilde{\rho }_{x}^{\pi } \\
&&+\rho _{x}^{\pi \prime }\left( \widetilde{\mathcal{K}}\right) ^{-1}\left( 
\frac{1}{\tau }\widetilde{\Omega }_{\tau }^{x}\right) \left( \widetilde{%
\mathcal{K}}^{\prime }\right) ^{-1}\widetilde{\rho }_{x}^{\pi }
\end{eqnarray*}

We now turn to the pricing of inflation options. Under the approximation (or
exact?) 
\begin{equation*}
\left. q_{t+\tau }-q_{t}\right\vert _{\mathcal{F}_{t}}\sim N\left( \tau
\cdot \mathcal{IE}_{t,\tau },\tau \cdot \mathcal{IU}_{t,\tau }\right) ,
\end{equation*}
we can derive the pricing formula for inflation caps and floors:%
\begin{eqnarray*}
P_{t,\tau ,K}^{CAP} &=&\exp \left( -\tau y_{t,\tau }^{N}\right) E_{t}^{%
\mathbb{\tilde{P}}}\left[ \left( \frac{Q_{t+\tau }}{Q_{t}}-\left( 1+K\right)
^{\tau }\right) ^{+}\right]  \\
&=&P_{t,\tau }^{N}\left[ 
\begin{array}{c}
e^{\tau \left( \mathcal{IE}_{t,\tau }+\frac{1}{2}\mathcal{IU}_{t,\tau
}\right) }\Phi \left( \frac{-\log \left( 1+K\right) +\left( \mathcal{IE}%
_{t,\tau }+\mathcal{IU}_{t,\tau }\right) }{\sqrt{\mathcal{IU}_{t,\tau }/\tau 
}}\right)  \\ 
-\left( 1+K\right) ^{\tau }\Phi \left( \frac{-\log \left( 1+K\right) +%
\mathcal{IE}_{t,\tau }}{\sqrt{\mathcal{IU}_{t,\tau }/\tau }}\right) 
\end{array}%
\right] 
\end{eqnarray*}%
and%
\begin{eqnarray*}
P_{t,\tau ,K}^{FLO} &=&\exp \left( -\tau y_{t,\tau }^{N}\right) E_{t}^{%
\mathbb{\tilde{P}}}\left[ \left( \left( 1+K\right) ^{\tau }-\frac{Q_{t+\tau }%
}{Q_{t}}\right) ^{+}\right]  \\
&=&P_{t,\tau }^{N}\left[ 
\begin{array}{c}
-e^{\tau \left( \mathcal{IE}_{t,\tau }+\frac{1}{2}\mathcal{IU}_{t,\tau
}\right) }\Phi \left( -\frac{-\log \left( 1+K\right) +\left( \mathcal{IE}%
_{t,\tau }+\mathcal{IU}_{t,\tau }\right) }{\sqrt{\mathcal{IU}_{t,\tau }/\tau 
}}\right)  \\ 
+\left( 1+K\right) ^{\tau }\Phi \left( -\frac{-\log \left( 1+K\right) +%
\mathcal{IE}_{t,\tau }}{\sqrt{\mathcal{IU}_{t,\tau }/\tau }}\right) 
\end{array}%
\right] 
\end{eqnarray*}%
Therefore,%
\begin{eqnarray}
p_{t,\tau ,K}^{CAP} &\equiv &\ln \left( P_{t,\tau ,K}^{CAP}\right)   \notag
\\
&=&-\tau \left[ a_{\tau }^{N}+b_{\tau }^{N\prime }x_{t}\right] +\ln \left( %
\left[ \exp \left( h_{0}\left( \tau ,x_{t}\right) \right) \Phi \left(
h_{1}\left( \tau ,x_{t}\right) \right) -\left( 1+K\right) ^{\tau }\Phi
\left( h_{2}\left( \tau ,x_{t}\right) \right) \right] \right) 
\end{eqnarray}%
where%
\begin{eqnarray*}
h_{0} &=&\tau \left( \mathcal{IE}_{t,\tau }+\frac{1}{2}\mathcal{IU}_{t,\tau
}\right)  \\
h_{1} &=&\frac{-\log \left( 1+K\right) +\left( \mathcal{IE}_{t,\tau }+%
\mathcal{IU}_{t,\tau }\right) }{\sqrt{\mathcal{IU}_{t,\tau }/\tau }} \\
h_{2} &=&\frac{-\log \left( 1+K\right) +\mathcal{IE}_{t,\tau }}{\sqrt{%
\mathcal{IU}_{t,\tau }/\tau }}
\end{eqnarray*}%
implying%
\begin{eqnarray*}
h_{0,x} &=&\tau \tilde{b}_{\tau }^{\pi } \\
h_{1,x} &=&\frac{1}{\sqrt{\mathcal{IU}_{t,\tau }/\tau }}\tau \tilde{b}_{\tau
}^{\pi } \\
h_{2,x} &=&\frac{1}{\sqrt{\mathcal{IU}_{t,\tau }/\tau }}\tau \tilde{b}_{\tau
}^{\pi }
\end{eqnarray*}

\end{document}
